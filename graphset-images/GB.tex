% command to generate figures
% pdflatex -shell-escape -synctex=1 -interaction=nonstopmode GB.tex
% 

\documentclass[class=minimal,border=1pt]{standalone}
% \documentclass{article}
\usepackage{tikz}
\usepackage{fontspec}
\usepackage{unicode-math}

\setmainfont{TeX Gyre Bonum}
\setsansfont{Lucida Sans}[Scale=MatchLowercase]
\setmonofont{Inconsolata}[Scale=MatchLowercase]



% \setmathfont[range={\mathup}]{Lucida Sans}
% \setmathfont[range={\mathit}]{Lucida Sans Italic}
%  \setmathfont[range=]

\usetikzlibrary{calc,matrix,shapes,positioning,fadings}

\tikzfading[name=fade out,
inner color=transparent!0,
outer color=transparent!100]


\newcommand{\convexpath}[2]{
[   
    create hullnodes/.code={
        \global\edef\namelist{#1}
        \foreach [count=\counter] \nodename in \namelist {
            \global\edef\numberofnodes{\counter}
            \node at (\nodename) [draw=none,name=hullnode\counter] {};
        }
        \node at (hullnode\numberofnodes) [name=hullnode0,draw=none] {};
        \pgfmathtruncatemacro\lastnumber{\numberofnodes+1}
        \node at (hullnode1) [name=hullnode\lastnumber,draw=none] {};
    },
    create hullnodes
]
($(hullnode1)!#2!-90:(hullnode0)$)
\foreach [
    evaluate=\currentnode as \previousnode using \currentnode-1,
    evaluate=\currentnode as \nextnode using \currentnode+1
    ] \currentnode in {1,...,\numberofnodes} {
-- ($(hullnode\currentnode)!#2!-90:(hullnode\previousnode)$)
  let \p1 = ($(hullnode\currentnode)!#2!-90:(hullnode\previousnode) - (hullnode\currentnode)$),
    \n1 = {atan2(\y1,\x1)}, 
    \p2 = ($(hullnode\currentnode)!#2!90:(hullnode\nextnode) - (hullnode\currentnode)$),
    \n2 = {atan2(\y2,\x2)},
    \n{delta} = {-Mod(\n1-\n2,360)}
  in 
    {arc [start angle=\n1, delta angle=\n{delta}, radius=#2]}
}
-- cycle
}



% Load the library
\usetikzlibrary{graphs, graphdrawing, external}
\usegdlibrary{circular}
\tikzset{external/system call={lualatex \tikzexternalcheckshellescape -output-format=dvi -halt-on-error -interaction=batchmode -jobname "\image" "\texsource"}}
\tikzexternalize
\begin{document}

% Figure 0 - Basic Figure with no hidden objects 
\begin{tikzpicture}[hn/.style={draw,circle,gray},
vn/.style={draw,circle, thick},
ve/.style={red,very thick}, 
he/.style={gray, thick}
]
\graph [
nodes=vn,
edges=ve,
simple necklace layout, 
] { 
1, 2, 3, 4, 5, 6, 
{
1 -- {2, 3, 6},
2 -- {3, 4},
3 -- {4, 6},
}
}; 
\end{tikzpicture}

% Figure 0 - Basic Figure with 6 hidden
\begin{tikzpicture}[hn/.style={draw,circle,gray},
vn/.style={draw,circle, thick},
ve/.style={red,very thick}, 
he/.style={gray, thick}
]
\graph [
nodes=vn,
edges=ve,
simple necklace layout, 
] { 
1, 2, 3, 4,  5, 6 [hn,source edge style=he,target edge style=he], 
{
1 -- {2, 3, 6},
2 -- {3, 4},
3 -- {4, 6},
}
}; 
\end{tikzpicture}

% Figure 0 - Basic Figure with 5 hidden
\begin{tikzpicture}[hn/.style={draw,circle,gray},
vn/.style={draw,circle, thick},
ve/.style={red,very thick}, 
he/.style={gray, thick}
]
\graph [
nodes=vn,
edges=ve,
simple necklace layout, 
] { 
1, 2, 3, 4,  
5 [hn,source edge style=he,target edge style=he], 
6,   
{
1 -- {2, 3, 6},
2 -- {3, 4},
3 -- {4, 6},
}
}; 
\end{tikzpicture}

% Figure 0 - Basic Figure with 6 hidden
\begin{tikzpicture}[hn/.style={draw,circle,gray},
vn/.style={draw,circle, thick},
ve/.style={red,very thick}, 
he/.style={gray, thick}
]
\graph [
nodes=vn,
edges=ve,
simple necklace layout, 
] { 
1, 2, 3, 4,  
5 [hn,source edge style=he,target edge style=he],  
6 [hn,source edge style=he,target edge style=he], 
{
1 -- {2, 3, 6},
2 -- {3, 4},
3 -- {4, 6},
}
}; 
\end{tikzpicture}

% Figure 0 - Basic Figure with 6 hidden
\begin{tikzpicture}[hn/.style={draw,circle,gray},
vn/.style={draw,circle, thick},
ve/.style={red,very thick}, 
he/.style={gray, thick}
]
\graph [
nodes=vn,
edges=ve,
simple necklace layout, 
] { 
1, 2, 3, 4,   5,  6, 
{
1 -- {2, 3, 6},
2 -- {3, 4},
3 -- {4, 6},
}
}; 
\draw[violet,thick] \convexpath{1,4,3,2}{0.4cm};
\end{tikzpicture}

\end{document}
